\section{Components in series and parallel}

Series resistances add:

\[ R_s = R_1 + R_2 + R_3 + \cdots \]

Parallel resistances diminish:

\[ R_p = \frac{1}{ \frac{1}{R_1} + \frac{1}{R_2} + \frac{1}{R_3} + \cdots} \]

Impedances behave like resistances.\\

Series capacitances diminish:

\[ C_s = \frac{1}{ \frac{1}{C_1} + \frac{1}{C_2} + \frac{1}{C_3} + \cdots} \]

Parallel capacitances add:

\[ C_p = C_1 + C_2 + C_3 + \cdots \]

Series inductances add:

\[ L_s = L_1 + L_2 + L_3 + \cdots \]

Parallel inductances diminish:

\[ L_p = \frac{1}{ \frac{1}{L_1} + \frac{1}{L_2} + \frac{1}{L_3} + \cdots} \]

For the special case of two components of the diminshing type (x being a dummy variable):

\[ \frac{1}{ \frac{1}{x_1} + \frac{1}{x_2} } = \frac{x_1 \cdot x_2}{x_1 + x_2} \]

\section{State devices / energy storage devices}
This section assumes time-invariant devices, i.e. capacitance/inductance is a fixed value, and not a function of time.

\subsection{Capacitors}

The current through a capacitor is a function of the \emph{rate of change} of voltage:

\[ i(t) = C \frac{dv(t)}{dt} \]

To find the voltage over a capacitor, we need to know its full history:

\[ v(t) = \frac{1}{C} \int_{-\infty}^{t} i(t) dt \]

... or we can simply do it by knowing the current through it between $t_1$ and $t_2$, plus the initial voltage:

\[ v(t_2) = \frac{1}{C} \int_{t_1}^{t_2} i(t) dt + v(t_1) \]

The energy stored in a capacitor is:

\[ E = \frac{1}{2} C v^2 \]

\subsection{Inductors}

The voltage over an inductor is a function of the \emph{rate of change} of current:

\[ v(t) = L \frac{di(t)}{dt} \]

To find the current through an inductor, we need to know its full history:

\[ i(t) = \frac{1}{L} \int_{-\infty}^t v(t) dt \]

... or we can simply do it by knowing the voltage over it between $t_1$ and $t_2$, plus the initial current through it:

\[ i(t_2) = \frac{1}{L} \int_{t_1}^{t_2} v(t) dt + i(t_1) \]

The energy stored in an inductor is:

\[ E = \frac{1}{2} L i^2 \]

\section{First order circuits}

General equation for an increasing/decaying exponential in a RC/RL circuit:

\[ v = V_S + (V_0 - V_S) e^{-t/RC} \]
\[ i = \frac{V_S}{R} + (i_0 - \frac{V_S}{R}) e^{-Rt/L} \]

where v = the voltage across the capacitor. For RL circuits, replace $V_S$ by the maximum current and $V_0$ by the initial current through the inductor. Above is for a thevenin equivalent circuit.\\

ZIR / Zero Input Response, same as above with $V_S = 0$:

\[ V_0 e^{-t/RC} \]

ZSR / Zero State Response, same as above with $V_0 = 0$:

\[ V_S(1 - e^{-t/RC}) \]

\newpage

\section{Second order circuits, impedance, filters}

Canonical form of the characteristic equation for second-order circuits; use this to match up the values of $\alpha$ and $\omega_0$ for a circuit:

\[ s^2 + 2\alpha s + {\omega_0}^2 = 0 \]

\subsection{For all LC and RLC circuits:}

\[ \text{Natural/undamped resonant radian frequency: } \omega_0 \text{ (rad/s)} \]
\[ \text{Damping factor: } \alpha \text{ (rad/s)} \]

Note that zeta ($\zeta$) is used as a damping factor in many texts; it is defined as
\[ \zeta = \frac{\alpha}{\omega_0} \text{ (dimensionless)} \]

The bandwidth $\Delta \omega$, i.e. the \emph{width} of the frequency \emph{band} that is above $\displaystyle \frac{1}{\sqrt{2}}$ times the input amplitude, is given by

\[ \Delta \omega = 2 \alpha \text{ (rad/s) (measured at } \frac{1}{\sqrt{2}} \text{ points)} \]

\[ \text{Quality factor: } Q = \frac{\omega_0}{2\alpha} \text{ (dimensionless)} \]

RLC circuits can be underdamped, overdamped, or critically damped.

\[ \text{Underdamped: } \omega_0 > \alpha \text{ or, equivalently, } Q > \frac{1}{2} \text { or, equivalently, } \zeta < 1 \]
\[ \text{Overdamped: } \omega_0 < \alpha \text{ or, equivalently, } Q < \frac{1}{2} \text { or, equivalently, } \zeta > 1 \]
\[ \text{Critically damped: } \omega_0 = \alpha \text{ or, equivalently, } Q = \frac{1}{2} \text { or, equivalently, } \zeta = 1 \]

When they are \emph{underdamped}, the \emph{damped resonant frequency} $\omega_d$ is given by
\[ \omega_d = \sqrt{{\omega_0}^2 - \alpha^2} \]
The naming here might be confusing; the \emph{damped} frequency is used for \emph{underdamped} systems. The reason is that the \emph{undamped} frequency is used for systems with no damping whatsoever, i.e. LC circuits with no resistor.\\
Underdamped RLC circuits are the only kind that oscillate, so the natural frequency is less interesting for overdamped circuits.

\subsection{Series RLC circuits}

\[ \omega_0 = \frac{1}{\sqrt{LC}} \text{ rad/s} \]
\[ f_0 = \frac{1}{2\pi \sqrt{LC}} \text { Hz} \]
\[ \alpha = \frac{R}{2L} \text{ rad/s} \]
\[ \Delta \omega = 2\alpha = \frac{R}{L} \text { rad/s} \]
\[ \text{Period: } \frac{2\pi}{\omega_0} = 2\pi \sqrt{LC} \text { seconds} \]
\[ Q = \frac{\omega_0}{2\alpha} = \frac{L}{R\sqrt{LC}} \text { (dimensionless)} \]

\subsection{Parallel RLC circuits}

\[ \omega_0 = \frac{1}{\sqrt{LC}} \text{ rad/s} \]
\[ f_0 = \frac{1}{2\pi \sqrt{LC}} \text { Hz} \]
\[ \alpha = \frac{1}{2RC} \text{ rad/s} \]
\[ \Delta \omega = 2\alpha = \frac{1}{RC} \text { rad/s} \]
\[ \text{Period: } \frac{2\pi}{\omega_0} = 2\pi \sqrt{LC} \text { seconds} \]
\[ Q = \frac{\omega_0}{2\alpha} = \frac{RC}{\sqrt{LC}} \text { (dimensionless)} \]

\subsection{Frequency- to time-domain conversion}
You can find the time-domain behavior of a circuit to sinusoidal input from nothing but a complex amplitude of the form $V_x$:
\[ v_X(t) = |V_x| \cos{(\omega t + \angle V_x)} \]
See below for information about how to calculate the magnitude $|z|$ and the angle $\angle z$ of a complex number.

\subsection{Complex algebra}
A few properties of complex numbers that are necessary to know:

\[ |a + jb| = \sqrt{a^2 + b^2} \]
\[ \angle (a + jb) = \arctan{(\frac{b}{a})} \text{ or, preferably, } \text{atan2}(a, b) \]

\[ |a + j0| = a \text { if $a > 0$; otherwise, the magnitude is just the absolute value } |a| \]
\[ |0 + jb| = b \]
\[ |0 - jb| = b \]

\[ \angle (a + j0) = 0 \]
\[ \angle (0 + jb) = \frac{\pi}{2} \]
\[ \angle (0 - jb) = -\frac{\pi}{2} \]

\[ |z_1 \cdot z_2| = |z_1| \cdot |z_2| \]
\[ \left| \frac{z_1}{z_2} \right| = \frac{|z_1|}{|z_2|} \]

\[ \angle (z_1 \cdot z_2) = \angle z_1 + \angle z_2 \]
\[ \angle \left( \frac{z_1}{z_2} \right) = \angle z_1 - \angle z_2 \]

\subsection{Impedances}

\[ \text{Resistor: } Z_R = R \]
\[ \text{Capacitor: } Z_C = \frac{1}{j\omega C} \]
\[ \text{Inductor: } Z_L = j\omega L \]

%%%%%%%%%%%%%%%%%%%%%%%%
%%% Circuit analysis %%%
%%%%%%%%%%%%%%%%%%%%%%%%

\chapter{Circuit analysis}

\section{Thevenin equivalent circuits}

Say we have an capacitor circuit to analyze:

\begin{figure} \begin{lateximage} \begin{circuitikz}[scale=1.2]
	\draw (0,0) node [ground] {} to [V=$V_S$] (0,3)
	to [R=$R_1$]  (3,3)
	to [R=$R_2$]  (6,3)
	to [C=$C$, v=$v_C$]   (6,0);
	
	\draw (3,3)                  to [R=$R_3$]  (3,0);
	\draw (0,0)                  to           (6,0);
\end{circuitikz} \end{lateximage} \caption{Capacitor Circuit} \end{figure}

Since this is a linear network, we can simplify it by calculating its \emph{Thevenin equivalent}. Consider the network as seen from the port where the capacitor is attached:

\begin{figure} \begin{lateximage} \begin{circuitikz}[scale=1.2]
	\draw (0,0) node [ground] {} to [V=$V_S$] (0,3)
	to [R=$R_1$]  (3,3)
	to [R=$R_2$, -o]  (6,3);
	
	\draw (3,3)                  to [R=$R_3$]  (3,0);
	\draw (0,0)                  to [short, -o]         (6,0);
	\draw (6,0)                  to [open, v>=$V_{TH}$] (6,3);
\end{circuitikz} \end{lateximage} \caption{Thevenin Equivalent} \end{figure}

$V_{TH}$, the open circuit voltage, will be given by the voltage divider formed by $R_3$ and $R_1$:

\[ V_{TH} = \frac{R_3}{R_1 + R_3} \cdot V_S \]

Since no current flows at the port (for the \emph{open circuit} voltage!), $R_2$ doesn't contribute at all.

We also want to measure the resistance ``looking in'' to this port; this will be the Thevenin resistance $R_{TH}$. To do this, we turn off all \emph{independant} voltage and current sources, by replacing all current sources with \emph{opens} and all voltage sources with \emph{short circuits}.\\
Leave dependant sources in the circuit!

\begin{figure} \begin{lateximage} \begin{circuitikz}[scale=1.2]
	\draw (0,0)                  to [short] (0,3)
	to [R=$R_1$]  (3,3)
	to [R=$R_2$, -o]  (6,3);
	
	\draw (3,3)                  to [R=$R_3$]  (3,0);
	\draw (0,0)                  to [short, -o]         (6,0);
	\draw (6,0)                  to [open] (6,3);
\end{circuitikz} \end{lateximage} \caption{Thevenin Equivalent II} \end{figure}

\[ R_{TH} = R_2 + (R_1 || R_3) = R_2 + \frac{R_1 \cdot R_3}{R_1 + R_3} \]

Now that we know the Thevenin voltage $V_{TH}$ and the Thevenin resistance $R_{TH}$, we can replace the circuit with a voltage source of voltage $V_{TH}$ volts in series with a resistor of value $R_{TH}$ ohm, and place the capacitor back into the circuit:

\begin{figure} \begin{lateximage} \begin{circuitikz}[scale=1.2]
	\draw (0,0) node [ground] {} to [V=$V_{TH}$] (0,3)
	to [R=$R_{TH}$] (3,3)
	to [C=$C$, v=$v_C$]      (3,0);
	\draw (0,0)                  to              (3,0);
\end{circuitikz} \end{lateximage} \caption{Thevenin Equivalent III} \end{figure}

Our previous circuit has now turned into a simple RC circuit, which is easier to analyze. See the chapter on RC circuits.\\

As a side note, another way of measuring the Thevenin resistance is to short circuit the output node, calculate/measure the short-circuit current (with all sources left intact, of course), and calculate $R_{TH}$ as $\frac{V_{TH}}{I_{SC}}$.\\

In summary:

$\bullet$ Calculate/measure the open circuit voltage $V_{TH}$ at the port\\
$\bullet$ Turn off all independent sources (make short circuits of voltage sources, and open circuits of current sources), but leave dependant sources intact\\
$\bullet$ Calculate/measure the resistance $R_{TH}$ at the port terminal pair\\
$\bullet$ Replace the original circuit with a series circuit of a voltage source (voltage $V_{TH}$ volts), a resistor (resistance $R_{TH}$ ohm) and the element you want to analyze.

\newpage

\section{Norton equivalent circuits}
Nortan equivalent circuits are very similar to Thevenin equivalents, but use a \emph{current source} in \emph{parallel} with a resistor rather than a \emph{voltage source} in \emph{series} with a resistor.

To convert a circuit to its Norton equivalent:

$\bullet$ Calcurate/measure the \emph{short circuit current}, i.e. the current that would flow through the output port if we were to short-circuit it. The result is the Norton current $I_N$.\\
$\bullet$ Turn off all independent sources (make short circuits of voltage sources, and open circuits of current sources), but leave dependant sources intact.\\
$\bullet$ Calculate/measure the resistance at the port terminal pair; the result is the Norton resistance $R_N$.\\
$\bullet$ Replace the original circuit with a parallel circuit of a current source (current $I_N$ amperes), a resistor (resistance $R_N$ ohm) and the element you want to analyze.\\

\begin{figure} \begin{lateximage} \begin{circuitikz}[scale=1.2]
	\draw (0,0) node [ground] {} to [I=$I_N$] (0,3);
	\draw (0,3) to (3,3);
	\draw (3,3) to [R=$R_N$] (3,0);
	\draw (6,3) to [C=$C$, v=$v_C$] (6,0);
	\draw (6,0) to (0,0);
	\draw (6,3) to (3,3);
\end{circuitikz} \end{lateximage} \caption{Norton Circuit} \end{figure}

Note that since the method for calculating the equivalent resistance is identical for the Thevenin and Norton methods, $R_N = R_{TH}$.
It is easy to convert between one and the other:

\[ R_N = R_{TH} \]
\[ I_N = \frac{V_{TH}}{R_{TH}} \]
\[ V_{TH} = I_N \cdot R_N \]