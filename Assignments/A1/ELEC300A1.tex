
%% Creator David Li
% Modified matlab xsl latex template file to suit needs
% This LaTeX was auto-generated from an M-file by MATLAB.
% To make changes, update the M-file and republish this document.

\documentclass[12pt]{scrartcl}
\usepackage[utf8]{inputenc}
%\usepackage[
%HomeHTMLFilename=index,     % Filename of the homepage.
%HTMLFilename={elec340-},       % Filename prefix of other pages.
%IndexLanguage=english,      % Language for xindy index, glossary.%
%latexmk,                    % Use latexmk to compile.
%%   OSWindows,                  % Force Windows. (Usually automatic.)
%mathjax,                    % Use MathJax to display math.
%]{lwarp}
\usepackage{graphicx}
\usepackage{color}
\usepackage{xcolor}
\usepackage{amsmath}
\usepackage[ocgcolorlinks]{hyperref}
\usepackage{bookmark}
%\usepackage[hmargin=2cm,vmargin=2.5cm]{geometry}
\usepackage{booktabs}
\usepackage[american,siunitx]{circuitikz}


\newcommand{\equal}{=}
\begin{document}

\begin{center}
	\hrule
	\vspace{.4cm}
	{\textbf { \large ELEC 300 --- Electric Circuits II}}
\end{center}
{\textbf{Name:}\ David Li \hspace{\fill} \textbf{Student Number:}} \ V00818631  \\
{\textbf{Due Date:} Monday, 11 January 2018, 11:30 AM \hspace{\fill} \textbf{Assignment:} Number 1 \\
	\hrule
	
	
	\paragraph{Problem 1}
Find out how much power the 2 A source delivers to the circuit.

\begin{circuitikz}
	\draw (0,0) to[I,i>=$2\si{\ampere}$] (0,2) %% Note _>= instead of >=
	to [short,-*] (2,2)
	to [R=$50\si{\ohm}$,v=$v_o$](2,0);
	\draw (2,2) to [R,l^=$1\si{\ohm}$] (4,2)
	to [V,v<=$45 \si{\volt}$](4,0)
	to [R,l^=$4\si{\ohm}$] (2,0);
	\draw (0,0) to [short,-*] (2,0);
\end{circuitikz}

Using node analysis:

\[
-2 = \frac{v_o}{50} + \frac{v_o-45}{1+4}=0 \rightarrow v_o = 50 \si{\volt}
\]
The power generated by the $2 \si{\ampere}$ source is $v_o \times 2 \si{\ampere} = 100 \si{\watt}$

\paragraph{Problem 2}
\begin{circuitikz}[american voltages]
	% Voltage source
	\draw (0,0) to [V, v>=$20 \si{\volt}$](0, 6) to (6, 6)
	% Left half bridge
	to [R, l_=$2 \si{k\ohm}$, *-*] (3,3) % Top left resistor
	to [R, l_=$5 \si{k\ohm}$, -*] (6,0);  % Bottom left resistor
	% Right half bridge
	\draw (6,6)
	to [R, l_=$1 \si{k\ohm}$, -*] (9, 3) % Top right resistor
	to [R, l_=$30 \si{k\ohm}$, -*] (6,0)  % Bottom left resistor
	% Draw connection to (-) terminal of voltage source
	to (6, 0) to (0,0);
	
	% Draw voltmeter
	\draw (3, 3) to [R, l_=$5 \si{k\ohm}$,i_=$i_o$, -*] (9, 3);
\end{circuitikz}

Using node analysis: and solving for  $v_1$ and $v_2$.

\begin{align}
& \frac{v_1}{30 \ 000} + \frac{v_1-v_2}{5 \ 000} + \frac{v_1-20}{2 \ 000} = 0 \\
& \frac{v_2}{1 \ 000} + \frac{v_2-v_1}{5 \ 000} + \frac{v_2-20}{5 \ 000} =0 \\
& \begin{bmatrix}
22 & -6 \\
-1 & 7
\end{bmatrix}\begin{bmatrix}
v_1 \\ v_2
\end{bmatrix}=\begin{bmatrix}
300 \\ 20
\end{bmatrix} \notag
\end{align}

\paragraph{Problem 3}
Determine the Thevenin equivalent with respect to terminals a-b. 
\subparagraph{OPEN CIRCUIT}
\begin{circuitikz}
	\draw (0,2) to[V,v>=$100\si{\ampere}$] (0,0); %% Note _>= instead of >=
	\draw (0,2) to [R,l^=$2.5 \si{k\ohm}$,i_>=$i_1$,-*] (2,2)
	to [R=$625\si{\ohm}$,-*](2,0);
	\draw (2,2) to [short] (4,2);
	\draw (4,0)to [cI,l_=$10^{-3}v_2$] (4,2);
	\draw (4,0)
	to [R,l^=$4\si{\ohm}$] (2,0);
	\draw (0,0) to [short,-*] (2,0);
	
	\draw (4,0) to [short,-*] (8,0);
	\draw (8,2)to [cV,l_=$5000i_1$](8,0)
	to [short] (10,0);
	\draw (8,2)
	to [R,l^=$4\si{k \ohm}$] (10,2)
	to [R,l^=$6\si{k \ohm}$,v_>=$v_2$] (10,0)
	to [short, -*] (12,0)
	node[label={[font=\footnotesize]above:$b$}] {}
	;
	
	\draw (10,2) to [short, -*] (12,2)
	node[label={[font=\footnotesize]above:$a$}] {}
	;
\end{circuitikz}

\begin{align}
& \frac{v_2}{6000} + \frac{v_2-5000i_1}{4000}= 0 \\
& 100 -2500i_1 - 625(i_1+10^{-3}v_2)=0 \\
\end{align}

Solving for equations above leads to $V_{TH}=v_2=60 \si{\volt}$ and $i_2=0.02 A$.
\subparagraph{CLOSED CIRCUIT}
$v_2=0$, $i_{1}^{\prime}=\cfrac{100}{2500+625}=0.032 \si{\ampere}$ and $i_{sc}=\cfrac{5000}{4000}i_{1}^{\prime}=0.04 \si{\ampere}$.


$R_{TH}=\frac{60}{0.04}=1.5 \si{k\ohm}$
\hrule
\subparagraph*{Final Circuit}
\begin{circuitikz}
	\draw (0,2) to[V,l_=$V_{TH} \equal 60 \si{\volt}$] (0,0); %% Note _>= instead of >=
	\draw (0,2)to [R,l^=$R_{TH} \equal 1.5 \si{k\ohm}$,-o] (4,2)
	node[label={[font=\footnotesize]right:$a$}] {};
	\draw (0,0) to [short,-o] (4,0)
	node[label={[font=\footnotesize]right:$b$}] {};
\end{circuitikz}
\paragraph{Problem 4}
Find the terminal voltage $V_{ab}$ using superposition. \newline
\begin{circuitikz}
	\draw (0,2) to[V,l_=$4 \si{\volt}$] (0,0); %% Note _>= instead of >=
	\draw (0,2)to [R,l^=$10 \si{\ohm}$] (2,2);
	\draw (2,2)to [cV,l^=$3V_{ab}$] (4,2);
	\draw (4,0) to  [I,l^=$2\si{\ampere}$](4,2);
	\draw (4,2) to [short,-*] (6,2)
		node[label={[font=\footnotesize]right:$a$}] {};
	\draw (0,0) to [short,-*] (6,0)
		node[label={[font=\footnotesize]right:$b$}] {};
	\draw (6,2) to [open,v_>=$V_{ab}$] (6,0);
\end{circuitikz}
\paragraph{Problem 5}
Assume that the following measurements were made:
When a 20  $\si{\ohm}$ resistor is connected to the terminals a,b, the voltage $v_{ab}$ is measured as  $100\si{\volt}$.
When a 50 $\si{\ohm}$ resistor is connected to the terminals a,b, the voltage $v_{ab}$ is measured as $200\si{\volt}$.
Find the Thevenin equivalent of the network with respect to terminals a,b. 

\begin{align*}
& 100 = V_{TH}\frac{R_{TH}}{20+R_{TH}} \\
& 200 = V_{TH}\frac{R_{TH}}{50+R_{TH}} \\
\end{align*}

Solving the equations above, $V_{TH} = 600 \si{\volt}$ and $R_{TH} = 100 \si{\volt}$
\end{document}
    
